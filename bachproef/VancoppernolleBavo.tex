%===============================================================================
% LaTeX sjabloon voor de bachelorproef toegepaste informatica aan HOGENT
% Meer info op https://github.com/HoGentTIN/latex-hogent-report
%===============================================================================

\documentclass[dutch,dit,thesis]{hogentreport}
% TODO:
% - If necessary, replace the option `dit`' with your own department!
%   Valid entries are dbo, dbt, dgz, dit, dlo, dog, dsa, soa
% - If you write your thesis in English (remark: only possible after getting
%   explicit approval!), remove the option "dutch," or replace with "english".

\usepackage{lipsum} % For blind text, can be removed after adding actual content

%% Pictures to include in the text can be put in the graphics/ folder
\graphicspath{{../graphics/}}

%% For source code highlighting, requires pygments to be installed
%% Compile with the -shell-escape flag!
%\usepackage[chapter]{minted}
%% If you compile with the make_thesis.{bat,sh} script, use the following
%% import instead:
\usepackage[chapter,outputdir=../output]{minted}
\usemintedstyle{solarized-light}

%% Formatting for minted environments.
\setminted{%
    autogobble,
    frame=lines,
    breaklines,
    linenos,
    tabsize=4
}

%% Ensure the list of listings is in the table of contents
\renewcommand\listoflistingscaption{%
    \IfLanguageName{dutch}{Lijst van codefragmenten}{List of listings}
}
\renewcommand\listingscaption{%
    \IfLanguageName{dutch}{Codefragment}{Listing}
}
\renewcommand*\listoflistings{%
    \cleardoublepage\phantomsection\addcontentsline{toc}{chapter}{\listoflistingscaption}%
    \listof{listing}{\listoflistingscaption}%
}

% Other packages not already included can be imported here
\usepackage{morewrites}

\usepackage{float} % voor code block

\usepackage[acronym]{glossaries}

\makeglossaries

\newacronym{esg}{ESG}{Environmental, Social en Governance}
\newacronym{poc}{PoC}{Proof of Concept}
\newacronym{llm}{LLM}{Large Language Model}
\newacronym{rag}{RAG}{Retrieval-Augmented Generation}
\newacronym{mvo}{MVO}{maatschappelijk verantwoord ondernemen}
\newacronym{eu}{EU}{Europese Unie}
\newacronym{csrd}{CSRD}{Corporate Sustainability Reporting Directive}
\newacronym{esrs}{ESRS}{European Sustainability Reporting Standards}

%%---------- Document metadata -------------------------------------------------
% TODO: Replace this with your own information
\author{Vancoppernolle Bavo}
\supervisor{Mevr. N. Declercq}
\cosupervisor{Dhr. G. Gueli}
\title[Optionele ondertitel]%
    {Een vergelijkende studie van RAG-configuraties met variabele chunkingstrategieën, embedding modellen en LLM's, geoptimaliseerd voor ESG-vraagstukken}
\academicyear{\advance\year by -1 \the\year--\advance\year by 1 \the\year}
\examperiod{1}
\degreesought{\IfLanguageName{dutch}{Professionele bachelor in de toegepaste informatica}{Bachelor of applied computer science}}
\partialthesis{false} %% To display 'in partial fulfilment'
\institution{Turtle S.r.l.}

%% Add global exceptions to the hyphenation here
\hyphenation{back-slash}

%% The bibliography (style and settings are  found in hogentthesis.cls)
\addbibresource{bachproef.bib}            %% Bibliography file
\addbibresource{../voorstel/voorstel.bib} %% Bibliography research proposal
\defbibheading{bibempty}{}

%% Prevent empty pages for right-handed chapter starts in twoside mode
\renewcommand{\cleardoublepage}{\clearpage}

\renewcommand{\arraystretch}{1.2}

%% Content starts here.
\begin{document}

%---------- Front matter -------------------------------------------------------

\frontmatter

\hypersetup{pageanchor=false} %% Disable page numbering references
%% Render a Dutch outer title page if the main language is English
\IfLanguageName{english}{%
    %% If necessary, information can be changed here
    \degreesought{Professionele Bachelor toegepaste informatica}%
    \begin{otherlanguage}{dutch}%
       \maketitle%
    \end{otherlanguage}%
}{}

%% Generates title page content
\maketitle
\hypersetup{pageanchor=true}

\input{voorwoord}
\input{samenvatting}

%---------- Inhoud, lijst figuren, ... -----------------------------------------

\tableofcontents

% In a list of figures, the complete caption will be included. To prevent this,
% ALWAYS add a short description in the caption!
%
%  \caption[short description]{elaborate description}
%
% If you do, only the short description will be used in the list of figures

\listoffigures

\begin{figure}[!htb]
  \centering
  \includegraphics[width=\textwidth]{../graphics/gantt.png}
  \caption{Planning bachelorproef 2025--2026}
  \label{fig:gantt}
\end{figure}

% If you included tables and/or source code listings, uncomment the appropriate
% lines.
\listoftables

\listoflistings

% Als je een lijst van afkortingen of termen wil toevoegen, dan hoort die
% hier thuis. Gebruik bijvoorbeeld de ``glossaries'' package.
% https://www.overleaf.com/learn/latex/Glossaries
\clearpage

\printglossary[type=\acronymtype]

%---------- Kern ---------------------------------------------------------------

\mainmatter{}

% De eerste hoofdstukken van een bachelorproef zijn meestal een inleiding op
% het onderwerp, literatuurstudie en verantwoording methodologie.
% Aarzel niet om een meer beschrijvende titel aan deze hoofdstukken te geven of
% om bijvoorbeeld de inleiding en/of stand van zaken over meerdere hoofdstukken
% te verspreiden!

%%=============================================================================
%% Inleiding
%%=============================================================================

\chapter{\IfLanguageName{dutch}{Inleiding}{Introduction}}%
\label{ch:inleiding}

% De inleiding moet de lezer net genoeg informatie verschaffen om het onderwerp te begrijpen en in te zien waarom de onderzoeksvraag de moeite waard is om te onderzoeken. In de inleiding ga je literatuurverwijzingen beperken, zodat de tekst vlot leesbaar blijft. Je kan de inleiding verder onderverdelen in secties als dit de tekst verduidelijkt. Zaken die aan bod kunnen komen in de inleiding~\\autocite{Pollefliet2011}:

De hedendaagse bedrijfswereld ondergaat een fundamentele overgang waarbij de focus verschuift van enkel financiële winstgevendheid naar een breder kader van duurzaamheid en maatschappelijke verantwoordelijkheid. Dit kader wordt aangeduid als ESG: Environmental, Social, and Governance. Wat ooit begon als een vrijwillige deelname voor maatschappelijk verantwoord ondernemen, is vandaag de dag geëvolueerd naar een strikt juridisch gereguleerd landschap. Europese richtlijnen zoals de Corporate Sustainability Reporting Directive (CSRD) verplichten bedrijven tot een ongeëvenaarde mate van transparantie over hun impact op mens en milieu.

Voor organisaties zoals Turtle Srl brengt deze evolutie een beduidende administratieve en technologische werklast met zich mee. Bedrijven worden geconfronteerd met gedetailleerde ESG-vragenlijsten en de daarbij horende standaarden zoals de European Sustainability Reportin Standards (ESRS). Het correct beantwoorden van deze vragen vereist het doorzoeken van grote hoeveelheden ongestructureerde data. Hoewel Large Language Models (LLM's) een mogelijkheid bieden om dit proces te automatiseren, kampen ze met fundamentele tekortkomingen zoals hallucinaties en een gebrek aan domeinspecifieke bedrijfskennis.

Deze bachelorproef onderzoekt hoe Retrieval-Augmented Generation (RAG) kan ingezet worden om een betrouwbaar, transparant en performant RAG-configuratie te bekomen dat Turtle Srl kan helpen bij het ontwikkelen van hun PoC rond interne ESG-rapportage.

\section{\IfLanguageName{dutch}{Probleemstelling}{Problem Statement}}%
\label{sec:probleemstelling}

% Uit je probleemstelling moet duidelijk zijn dat je onderzoek een meerwaarde heeft voor een concrete doelgroep. De doelgroep moet goed gedefinieerd en afgelijnd zijn. Doelgroepen als ``bedrijven,'' ``KMO's'', systeembeheerders, enz.~zijn nog te vaag. Als je een lijstje kan maken van de personen/organisaties die een meerwaarde zullen vinden in deze bachelorproef (dit is eigenlijk je steekproefkader), dan is dat een indicatie dat de doelgroep goed gedefinieerd is. Dit kan een enkel bedrijf zijn of zelfs één persoon (je co-promotor/opdrachtgever).

De kern van het probleem situeert bevindt zich op het snijvlak van juridische compliance omtrent ESG en de technologische beperkingen van ‘stand-alone’ LLM’s. Turtle Srl moet voldoen aan steeds strengere rapportage-eisen, maar deze data handmatig verwerken, of met behulp van LLM’s is een tijdrovende en foutgevoelige taak.

De nood aan ESG-rapportage is niet langer optioneel. Het is een overlevingsvoorwaarde geworden in de mondiale supply chain. De EU, stakeholders en klanten eisen gedetailleerde ESG rapporten die transparant weergeven wat de impact op mens en milieu is. De structurele uitdaging hierbij is de fragmentatie en heterogeniteit van de data. Zo is de informatie vaak verspreid over verschillende databronnen en kan het verschillende dataformaten aannemen. Daarnaast is ook de juridisch bewijslast verzwaard, aangezien elke claim in een rapport terug moet te vinden zijn in de documenten van het bedrijf.

Hoewel LLM’s  sterke tekstuele vaardigheden hebben, zijn ze voor deze specifieke taak onvoldoende als 'stand-alone' oplossing. Een eerste probleem is de knowledge cut-off, waardoor ze niet op de hoogte zijn van de meest recente veranderingen omtrent ESG richtlijnen die na het pre-training proces worden gemaakt. Daarnaast is er het risico op hallucinaties wanneer ze niet over de juiste context beschikken, wat binnen een audit-context onaanvaardbaar is. Ten slotte bieden standaard LLM's geen automatische bronvermelding, wat de traceerbaarheid van de gegenereerde antwoorden bemoeilijkt.

De beoogde doelgroep van dit onderzoek is mijn stagebedrijf, Turtle Srl. De resultaten van deze studie zullen direct bijdragen aan de ontwikkeling van hun Proof-of-Concept (PoC) om aan ESG-rapportage te doen door middel van een RAG-systeem.

\section{\IfLanguageName{dutch}{Onderzoeksvraag}{Research question}}%
\label{sec:onderzoeksvraag}

Dit onderzoek vertrekt vanuit de volgende onderzoeksvraag:

\textit{Welke combinatie van chunking strategie, embedding model en LLM resulteert in een optimaal RAG-systeem voor Turtle Srl’s PoC?}

Om deze hoofdvraag te beantwoorden, worden de volgende deelvragen onderzocht:

\begin{itemize}
  \item	Waar komt de nood aan ESG-rapportage vandaan?
  \item	Welke structurele en juridische uitdagingen zijn er bij het beantwoorden van ESG-vragenlijsten (zoals EcoVadis en VSME)?
  \item	Welke tekortkomingen en uitdagingen hebben standaard LLM’s bij het beantwoorden van ESG questionnaires?
  \item Hoe beïnvloeden verschillende chunking strategieën de retrieval-kwaliteit van interne ESG-documenten?
  \item Welke functie heeft een embedding model binnen een RAG-pipeline?
  \item Wat is de taak van een LLM binnen de RAG-architectuur?
  \item Welke RAG framework is geschikt voor het aanmaken van aangepaste modules met behulp van open source-oplossingen?
  \item Hoe wordt een dynamische RAG pipeline gemaakt om combinaties van verschillende embedding modellen, chunking strategieën en LLM’s systematisch te vergelijken?
  \item Welke meetwaarden worden gebruikt om de variabele RAG-componenten te evalueren?
\end{itemize}

\section{\IfLanguageName{dutch}{Onderzoeksdoelstelling}{Research objective}}%
\label{sec:onderzoeksdoelstelling}

% Wat is het beoogde resultaat van je bachelorproef? Wat zijn de criteria voor succes? Beschrijf die zo concreet mogelijk. Gaat het bv.\ om een proof-of-concept, een prototype, een verslag met aanbevelingen, een vergelijkende studie, enz.

Het beoogde resultaat van deze bachelorproef is een vergelijkende studie en een technisch adviesrapport voor de realisatie van een Proof-of-Concept voor Turtle Srl.

De criteria voor succes zijn als volgt gedefinieerd:

\begin{itemize}
  \item Betrouwbaarheid: Het systeem moet een faithfulness-score (getrouwheid aan de bron) behalen die hallucinaties tot een minimum beperkt.
  \item Traceerbaarheid: Elk gegenereerd antwoord moet voorzien zijn van correcte source attribution, waarbij direct verwezen wordt naar de metadata (bestandsnaam en paginanummer) in de vector database.
  \item Efficiëntie: De gekozen configuratie moet een optimaal evenwicht bieden tussen de rekenkosten (token-verbruik) en de snelheid van antwoorden (latency). Aangezien dat, naast het voldoen aan de technische requirements, de PoC van Turtle Srl die zal volgen uit dit onderzoek ook business value moet zal moeten hebben.
  \item Modulariteit: De voorgestelde architectuur moet dynamisch zijn, wat betekent dat componenten (zoals een embedding model) vervangen kunnen worden zonder de gehele pipeline te herschrijven.
\end{itemize}

Uiteindelijk zal dit leiden tot een technisch ontwerp van een dynamische RAG-pipeline, gebouwd met Haystack 2.x, PocketFlow en Hypster, waarmee Turtle Srl hun ESG-workflows op een verantwoorde en schaalbare manier kan automatiseren.

\section{\IfLanguageName{dutch}{Opzet van deze bachelorproef}{Structure of this bachelor thesis}}%
\label{sec:opzet-bachelorproef}

% Het is gebruikelijk aan het einde van de inleiding een overzicht te
% geven van de opbouw van de rest van de tekst. Deze sectie bevat al een aanzet
% die je kan aanvullen/aanpassen in functie van je eigen tekst.

Na deze inleiding volgt de literatuurstudie in Hoofdstuk~\ref{ch:stand-van-zaken} waarin de theoretische fundamenten van ESG en RAG uiteen worden gezet.

Daarna volgt de methodologie in Hoofdstuk~\ref{ch:methodologie}. Hier wordt de experimentele opzet en evaluatie van de RAG-pipeline toegelicht, gebaseerd op het evaluatieframework van \autocite{Es2024}.

% TODO: Vul hier aan voor je eigen hoofstukken, één of twee zinnen per hoofdstuk

Tot slot wordt in Hoofdstuk~\ref{ch:conclusie} een conclusie gegeven en een antwoord geformuleerd op de onderzoeksvragen, en worden aanbevelingen geformuleerd voor zowel verder onderzoek als praktische implementatie. 
\chapter{\IfLanguageName{dutch}{Stand van zaken}{State of the art}}%
\label{ch:stand-van-zaken}

% Tip: Begin elk hoofdstuk met een paragraaf inleiding die beschrijft hoe
% dit hoofdstuk past binnen het geheel van de bachelorproef. Geef in het
% bijzonder aan wat de link is met het vorige en volgende hoofdstuk.

% Pas na deze inleidende paragraaf komt de eerste sectiehoofding.

Deze literatuurstudie een overzicht van de huidige stand van zaken voor dit onderzoek. Het probleemdomein betreft de nood aan \gls{esg}-rapportage en de uitdagingen die daarbij komen kijken. Aangezien de \gls{poc} waar dit onderzoek voor dient, zich richt op het beantwoorden van \gls{esg}-gerelateerde vragenlijsten. Daarnaast worden de tekortkomingen van \glspl{llm} besproken op vlak van \gls{esg}-rapportage. Het oplossingsdomein focust op \gls{rag} en hoe dit een oplossing kan zijn voor een betere rapportage op basis van interne bedrijfsdocumenten. Verder wordt er besproken hoe een dynamische \gls{rag} configuratie kan worden opgesteld om efficiënt de variabele componenten te implementeren en evalueren.

\section{ESG en ESG-rapportage}

\subsection{Evolutie van ESG}

In de hedendaagse bedrijfsvoering is de overgang van een primaire focus op winst naar duurzame waarde creatie onomkeerbaar geworden. In deze beweging staat het concept \gls{esg} centraal, wat staat voor Envorinmental, Social en Governance. De Environmental pilaar beoordeelt de inspanningen van een bedrijf om het milieu te beschermen \autocite{Radzi2023}. De Social pilaar beschrijft hoe een bedrijf omgaat met mensen en zakelijke relaties \autocite{Radzi2023}. En de Governance pilaar biedt richtlijnen voor het management van een bedrijf \autocite{Radzi2023}. Het dient als een kader om maatschappelijke inspanningen meetbaar en kwantificeerbaar te maken.

\subsection{Voordelen van ESG-rapportage}

De literatuur toont aan dat inspanningen leveren om verbeteringen te maken op vlak van \gls{esg} en het correct rapporteren, kan leiden tot verschillende voordelen voor bedrijven. \autocite{Aydogmus2022} onderzochten de impact van \gls{esg}-prestaties op de bedrijfswaarde en winstgevendheid en stelt vast dat \gls{esg}-scores een positieve en sterke relatie hebben met winstgevendheid. \textcite{Lew2024} bestudeerde het maatschappelijke belang van \gls{mvo}. Daaruit blijkt dat inspanningen in \gls{esg} bijdragen aan een beter imago van ondernemingen en zowel operationele als financiële risico’s verminderen \autocite{Lew2024}. Verdere studies tonen aan dat er een positief verband ligt tussen hogere \gls{esg}-scores van bedrijven en de toegang tot goedkopere financieringsbronnen en bedrijfsmiddelen \autocite{Gholami2022}. \textcite{Fornasari2024} noemt het openbaar van niet-financiële bedrijfsinformatie, bijvoorbeeld \gls{esg}: “een strategisch instrument om vertrouwen te kweken, risico's te beperken en de legitimiteit en het succes van bedrijven op lange termijn te waarborgen”. Aangezien de nadruk bij \gls{mvo} ligt op een nood aan transparantie en het voldoen aan maatschappelijke eisen \autocite{Fornasari2024}.

\subsection{Juridisch kader}

Een belangrijke drijfveer achter de evolutie van \gls{esg} is het wettelijke kader. Voorheen was niet-financiële rapportage gebaseerd op vrijwillig deelname \autocite{Centre.2023}. Maar om greenwashing tegen te gaan en het ontwikkelen van duurzame producten, heeft de \gls{eu} strikte wetgevingen opgesteld \autocite{Commission2020}. In 2022 heeft de \gls{eu} de \gls{csrd} aangenomen, dat gedetailleerde rapportagevereisten introduceert en verplicht voor bijna 50.000 ondernemingen \autocite{Fornasari2024,AccountancyEurope2024}. Met behulp van deze richtlijnen kunnen bedrijven hun \gls{esg}-informatie op een gestructureerde en vergelijkbare manier presenteren, met als doel een meer transparante en verantwoordelijke bedrijfsvoering \autocite{Darnall2022}. De \gls{csrd} vereist rapportage volgens de \gls{esrs}. Dit betekend dat de rapportage van bedrijven in een digitaal, machinaal leesbaar formaat moet worden opgeleverd \autocite{Centre.2023}. Verder wordt externe verificatie verplicht en wordt uitgevoerd door externe auditors \autocite{Centre.2023}. Deze strengere regelgeving zorgt voor een toename in de administratieve last en de complexiteit die bedrijven ervaren bij het beantwoorden van \gls{esg}-vragenlijsten.

\section{Uitdagingen ESG-rapportage}

De transitie naar een economie waar ESG een grotere rol speelt, heeft geleid tot de introductie van complexe wettelijke kaders en richtlijnen, zoals de CSRD en ESRS. Voor organisaties zoals Turtle Srl brengt dit structurele uitdagingen met zich mee.

\subsection{Datastructuur en fragmentatie}

Uit de literatuur blijkt dat een eerste bewezen struikelblok de heterogene aard van ESG-data is. ESG-informatie bestaat uit zowel gestructureerde data als ongestructureerde data \autocite{Lavin2021}. Gestructureerde data zijn gegevens die eenvoudig kunnen worden verwerkt via ETL-procedures, en zijn terug te vinden in databases en spreadsheets \autocite{Peng2024}. Ongestructureerde data, zoals documenten, pdf’s en e-mails, vormen vanwege hun niet-gestandaardiseerde formaten een uitdaging voor het uittrekken van belangrijke informatie \autocite{Peng2024,Morales2022}. \textcite{FundsEurope2025} rapporteert dat 55\% van de bedrijven uitdagingen verwacht op het gebied van datakwaliteit en consistentie bij CSRD-rapportage, en 45\% van de bedrijven zijn bezorgd over het feit of zij voldoende resources hebben om aan de richtlijnen te voldoen. Dit maakt het handmatig samenbrengen van deze gefragmenteerde bronnen tot een samenhangend rapport een foutgevoelig en tijdrovend proces \autocite{Gharpure2025}.

\subsection{Kosten- en tijdsintensiviteit}

Verder wordt in de literatuur besproken dat de bewijslast voor ESG-claims de afgelopen jaren sterk is toegenomen \autocite{Centre.2023,Lew2024}. Bedrijven moeten niet alleen rapporteren over hun eigen impact, maar vaak ook over die van hun volledige toeleveringsketen \autocite{Fornasari2024,Gharpure2025}. Daarnaast besteden organisaties jaarlijks gemiddelde 7.500 personeelsuren aan het proces van data-extractie, verificatie en het schrijven van duurzaamheidsrapporten \autocite{Gharpure2025}. Voor middelgrote ondernemingen vormen de administratieve last en de kosten die daarmee gepaard gaan een valkuil voor effectieve het streven naar een meer duurzaam beleid.

\subsection{Auditeerbaarheid en transparantie}

In tegenstelling tot financiële verslaglegging, die steunt op decennia aan gestandaardiseerde processen, is ESG-rapportering nog volop in ontwikkeling \autocite{Yadav2024}. Er is een strikte noodzaak voor traceerbaarheid. Elke bewering in een ESG-rapport moet gestaafd kunnen worden met brondocumentatie om beschuldigingen van greenwashing te voorkomen. Dit maakt de source attribution niet alleen een technische wens, maar een juridische en ethische noodzaak.

\section{Tekortkomingen van LLM's}

LLM’s zoals GPT-4 of Gemini zijn de afgelopen jaren sterk geëvolueerd. Ze worden ingezet voor het uitvoeren van uiteenlopende taken zoals het samenvatten van video-opnames, vertalen van teksten of het genereren van code. Toch schieten ze fundamenteel tekort voor de specifieke eisen van ESG-rapportage, waar feitelijke precisie, juridische traceerbaarheid en actualiteit onherroepelijk zijn \autocite{Alansari2025}.

\subsection{Inefficiëntie bij gefragmenteerde data}

Zoals eerder beschreven, is ESG-data sterk gefragmenteerd. Standaard LLM's zijn getraind op enorme hoeveelheden algemene tekst, maar missen de specifieke structuur om verbanden te leggen tussen gefragmenteerde bedrijfsdocumenten. Zonder een extern mechanisme, zoals de Retriever in een RAG-systeem, kan een LLM geen informatie ophalen uit specifieke PDF-bestanden of spreadsheets die niet in zijn trainingsset zaten \autocite{BerniakWozny2025}. Hierdoor blijft de unieke context van een bedrijf onbereikbaar voor het model.

\subsection{Hallucinaties}

In een domein waar cijfers en feiten de basis vormen voor juridische verantwoording, is het risico op hallucinaties van LLM's een risico. In een onderzoek van \autocite{Rahman2024}, waarin 6 LLM’s getest werden op een publieke dataset, blijkt dat de totale feitelijke hallucinatie tussen 59\% en 82\% ligt. Een model kan op basis van taalpatronen een zeer overtuigend antwoord formuleren, terwijl de onderliggende cijfers feitelijk onjuist zijn of gebaseerd zijn op een verkeerde interpretatie van de context. Voor ESG-rapportage is een "ongeveer correct" antwoord onvoldoende; elk cijfer moet exact zijn.

\subsection{Knowledge Cut-off}

ESG-standaarden en bedrijfsresultaten evolueren razendsnel. De CSRD-richtlijnen worden continu verfijnd en bedrijfsgegevens blijven toenemen en veranderen per kwartaal. Vanwege de knowledge cut-off is de interne kennis van een LLM statisch en per definitie verouderd zodra het model is getraind. Een model dat is getraind in 2024, heeft geen kennis van een duurzaamheidsrapport dat in 2025 is gepubliceerd \autocite{Dye2021}. Dit gebrek aan actualiteit maakt een standaard LLM ongeschikt voor real-time rapportage of auditing.

\section{RAG}

Om deze tekortkomingen op te lossen, biedt RAG, geïntroduceerd door \textcite{Lewis2020}, een veelbelovend architectuurpatroon. Het kernidee achter RAG is dat LLM’s, hoewel krachtig in het vastleggen van taalpatronen en algemene kennis tijdens pre-training, beperkt zijn in hun vermogen om actuele, domeinspecifieke of feitelijke informatie te onthouden en correct toe te passen. Door een externe kennisbron te combineren met een LLM, kunnen RAG-systemen relevante documenten ophalen en deze gebruiken als context voor het genereren van antwoorden \autocite{Lewis2020}. Door te verwijzen naar externe bronnen verhoogt de feitelijke correctheid van het model en vermindert de oplevering van onjuiste informatie \autocite{Gianluca2024}. Omdat een LLM niet meer moet vertrouwen op de parametrische kennis die tijdens de pre-training is opgeslagen, werkt die nu als een actieve verwerker van externe, niet parametrische data. Wanneer een gebruiker een vraag stelt, zoekt het systeem eerst in de externe database naar relevante datafragmenten. Deze fragmenten worden daarna samen met de oorspronkelijke vraag als context aan de LLM gegeven. Gegeven de noodzaak tot accurate antwoorden en correcte bronvermelding voor de casus van Turtle Srl, vormt de implementatie van een RAG-systeem de basis voor hun PoC.

\subsection{Werking standaard RAG-architectuur}

RAG koppelt een retriever aan een generator, waarbij semantisch relevante datafragmenten uit een externe kennisbank worden opgehaald. Deze dienen als context voor de generatie van nauwkeurige antwoorden \autocite{Lewis2020,Gianluca2024}.

Dit proces beschreven als een model dat twee componenten combineert. De retriever ($p_{\eta}(z|x)$) met parameters $\eta$ berekent een kansverdeling over tekstfragmenten ($z$) gegeven een vraag ($x$). Het resultaat is een top-K selectie van de meest relevante datafragmenten. De generator ($p_{\theta}(y_i|x, z, y_{1:i-1})$) met parameters $\theta$ genereert het huidige token ($y_i$) gebaseerd op de oorspronkelijke vraag ($x$), het opgehaalde fragment ($z$), en de reeds gegenereerde tokens ($y_{1:i-1}$) \autocite{Lewis2020}.

\begin{figure}
  \centering
  \includegraphics[width=0.8\textwidth]{RAG_component_overzicht.png}
  \caption[Werking standaard RAG architectuur]{\label{fig:rag_component_overview}Schematische weergave van het RAG-model waarbij relevante documenten uit een externe bron worden opgehaald om de tekstgeneratie te ondersteunen \autocite{Lewis2020}.}
\end{figure}

\subsection{Dynamische aanpak}

Voor het vergelijken van verschillende variabele componenten in een RAG-systeem, is een dynamische architectuur nodig die de variabiliteit van componenten ondersteunt.

\subsubsection{Component-gebaseerde architectuur}

Pipeline-orchestration binnen RAG verwijst naar het beheer van de gegevensstroom tussen de retriever, de generator en de tussenliggende verwerkingsstappen \autocite{Zhang2025GenAI}. In moderne architecturen is de verschuiving merkbaar van lineaire ketens naar graafgebaseerde workflows.

Voor dit onderzoek wordt Haystack 2.x ingezet. Waar eerdere versies van dergelijke frameworks vaak rigide waren, introduceert Haystack 2.x een architectuur gebaseerd op een gerichte graaf \autocite{HaystackPipelines2026}. Elke stap in de pipeline dient als een onafhankelijk component met strikt gedefinieerde in- en uitgangen (sockets). Deze dynamiek is cruciaal voor de ESG-casus van Turtle Srl, aangezien het toelaat om metadata (zoals paginanummers uit Qdrant) op een transparante manier door te geven aan de generation laag zonder de integriteit van de tekstfragmenten te verstoren \autocite{HaystackMetadataRouter2026}.

\subsubsection{Hyperparameterization}

Hyperparameterization in de context van RAG is het systematisch variëren van systeeminstellingen of componenten om de prestaties te optimaliseren. In dit onderzoek worden drie hoofdvariabelen geïdentificeerd, wat resulteert in een testmatrix van 27 unieke configuraties ($3^3$ indien men drie variabelen met drie opties hanteert).

In een bedrijfscontext zoals die van Turtle Srl is de meest accurate configuratie vaak ook de duurste of traagste. Hyperparameter-optimalisatie (HPO) is noodzakelijk om een superieure Pareto-front, de verzameling van de best mogelijke oplossingen in een probleem met meerdere requirements, te bereiken: de set configuraties die de best mogelijke balans bieden tussen tegenstrijdige doelstellingen zoals operationele kosten, latentie en feitelijke nauwkeurigheid \autocite{Barker2025}. Zonder systematische tuning riskeert men een systeem dat weliswaar accuraat is, maar door hoge inferentiekosten niet schaalbaar is voor grootschalige ESG-rapportage.

In dit onderzoek wordt deze dynamische aanpak ingevuld door de technische samenwerking tussen Hypster en PocketFlow. Hypster is een configuration framework dat het mogelijk maakt om verschillende hyperparameters - zoals variabel chunking, embeddings en LLM’s – te definiëren als injecteerbare variabelen. Dankzij deze parameters is het mogelijk om met 1 pipeline alle mogelijke combinaties van de variabele componenten te testen.

\begin{listing}[H]
  \begin{minted}{python}
    import os
    import llm
    from hypster import HP, instantiate

    def llm_config(hp: HP):
    model_name = hp.select(["gpt-4o-mini", "gpt-4o"], name="model_name")
    temperature = hp.float(0.0, name="temperature", min=0.0, max=1.0)
    max_tokens = hp.int(256, name="max_tokens", max=2048)

    return {
        "model_name": model_name,
        "temperature": temperature,
        "max_tokens": max_tokens
    }

  \end{minted}
  \caption[Voorbeeld codefragment HyPSTER - variabele LLM's]{Een voorbeeld van de syntax om meerdere LLM's, hier gpt-4o-mini en gpt-4o, te definiëren in een variabele met HyPSTER. Bron: https://github.com/gilad-rubin/hypster/blob/master/docs/getting-started/usage-examples/llms-and-generative-ai.md}
\end{listing}

PocketFlow is een flow-orchestrator dat de experimentele loop zal coördineren. Het framework garandeert dat de 27 combinaties systematisch worden getoetst op hun vermogen om "kennisconflicten" te verminderen — situaties waarin geëxtraheerde documenten botsen met de interne kennis van het LLM — terwijl de inferentiekosten beheersbaar blijven \autocite{Fu2024}.

\subsubsection{Chunking}

Het eerste component in een RAG-pipeline is chunking. Chunking is de techniek waarbij documenten worden opgesplitst in kleinere tekstdelen, die chunks worden genoemd. De manier waarop documenten worden opgedeeld, bepaalt welke informatie kan worden opgehaald en hoe efficiënt de LLM relevante antwoorden kan genereren met deze informatie \autocite{Moro2025}.

De literatuur toont dat chunking een belangrijke rol speelt de pipeline. Een eerste factor is het limiet op de hoeveelheid tokens die een LLM kan verwerken. Zonder chunking zou een LLM, vanwege een beperkt context window, onmogelijk grote documenten in het kunnen geheugen opslaan. Daarnaast zijn zoekopdrachten in vectordatabases effectiever wanneer ze worden uitgevoerd op kleinere datafragmenten. Te grote chunks veroorzaken ruis door irrelevante informatie te groeperen met een relevant fragment uit een document. Dit zorgt ervoor dat het embedding-model minder goed in staat is om de relevante context van een chunk te herkennen. Bovendien verlaagd chunking de rekenkosten en latency, door enkel de belangrijkste chunks naar de LLM te sturen.

Bij het opstellen van een chunking-strategie zijn er 3 parameters van belang:

\begin{itemize}
  \item \textbf{Chunk size} is een numerieke waarde die aangeeft hoeveel tekens, of tokens, er 1 chunk worden opgenomen. Een kleinere chunk size zorgt voor meer granulaire chunks, terwijl een grotere chunk size meer context per chunk behoudt \autocite{Guenther2024}.
  \item \textbf{Chunk overlap} is mate waarin opeenvolgende chunks elkaar mogen overlappen om de continuïteit van de context tussen chunks te waarborgen \autocite{GomezCabello2025}.
  \item \textbf{Context length} is de totale capaciteit van de LLM om informatie te verwerken. De chunking-strategie moet zo worden gekozen dat de opgehaalde chunks (top-k resultaten) de contextlengte van het LLM niet overschrijden.
\end{itemize}

\subsubsection{Vector embeddings en vector database}

Embedding modellen transformeren tekst naar numerieke vectoren (vector embeddings) die in een hoog-dimensionale ruimte worden opgeslagen. De essentie van RAG is gebaseerd op de veronderstelling dat semantisch gelijke teksten dicht bij elkaar liggen in deze ruimte.

De pipeline die verder in dit onderzoek wordt opgezet, gebruikt text-embedding-3-large (OpenAI), BGE-M3 (BAAI) en multilingual-e5-large-instruct (intfloat) als variabele embeddingmodellen.

De daaruit volgende embeddings worden opgeslagen in een vector database. De vector database een cruciaal element voor de efficiënte opslag en toegankelijkheid van semantische informatie \autocite{Pawlik2025}. Het is een gespecialiseerd systeem dat specifiek is ontworpen om hoog-dimensionale vectordata te verwerken, indexeren, vergelijken en op te slaan. Tijdens een zoekopdracht wordt de gegenereerde query-vector vergeleken met deze opgeslagen vectoren om de meest relevante informatie te identificeren en in een zo kort mogelijke tijd op te halen \autocite{Rahul2024}.

De vector database heeft ook de mogelijkheid om metadata toe te voegen aan de vector embeddings. Dit gebeurt aan de hand van labels die toegekend worden aan de vectoren en samen worden opgeslagen. Deze labels bevatten data zoals een datum, categorie, user-ID of taal.

\subsubsection{LLM}

Het generatieve component, het LLM, gebruikt de context uit de vector database om samen met de gestelde vraag een coherent antwoord te formuleren. Concreet bouwt de LLM binnen RAG voort op zijn in‑context learning‑capaciteiten: de opgehaalde ESG‑fragmenten fungeren als uitgebreide prompt, waardoor het model tijdens generatie kan “redeneren over bewijs” in plaats van te vertrouwen op verouderde of onvolledige trainingskennis \autocite{Wang2025}. Studies tonen dat dit hybridemodel 20\% tot 30\% betere prestaties oplevert op knowledge‑intensive vraag-en-antwoordtaken dan pure seq2seq‑modellen \autocite{Lewis2020}. RAG overtreft bijvoorbeeld BART en T5 op Natural Questions, WebQuestions en TriviaQA, terwijl de gegenereerde antwoorden specifieker zijn, en feitelijk meer correct \autocite{Lewis2020}.

%\begin{itemize}
%  \item \textbf{gpt-4o} 
%  \item \textbf{gemini-2.5-flash} 
%  \item \textbf{Mistral Large 2.1} 
%\end{itemize}

\section{Evaluatie RAG}

In dit deel van de literatuurstudie wordt de stand van zaken rond de evaluatie van RAG besproken, met bijzondere aandacht voor domeinspecifieke use cases zoals ESG‑rapportage. De evaluatie van een RAG-systeem teruggeeft hangt immers af van twee afzonderlijke processen — retrieval en generatie — waardoor verouderde NLP‑metrics (zoals BLEU of ROUGE) slechts een beperkt beeld geven van de werkelijke kwaliteit \autocite{Yu2024}. Voor de Proof of Concept bij Turtle Srl, waar verschillende combinaties van chunking, embedding‑modellen en LLM’s op een vaste ESG‑vragenlijst worden geëvalueerd, is een gericht evaluatiekader nodig dat zowel inhoudelijke kwaliteit (faithfulness, answer relevancy) als operationele aspecten (latency, kosten) en juridische vereisten (source attribution) dekt.

\subsection{Evolutie van het evalueren}

De evaluatie van taalmodellen is historisch geëvolueerd van n‑gram‑gebaseerde metrics naar semantische en taak‑specifieke benchmarks. N-gram-gebaseerde evaluatie, zoals BLEU en ROUGE, is een evaluatiemethode die de kwaliteit van gegenereerde tekst beoordeelt door de overlap te meten tussen de modeloutput en een referentietekst. Deze vorm van evaluatie is echter beperkt in bruikbaarheid tot het evalueren van machinevertalingen en samenvattingen.

Met de opkomst van transformer‑gebaseerde modellen werden semantische metrics zoals BERTScore geïntroduceerd, die contextuele embeddings gebruiken om de semantische gelijkenis tussen output en referentie te schatten \autocite{Devlin2019}. Daarnaast ontstond ook MTEB (Massive Text Embedding Benchmark). Een benchmark dat embedding‑modellen evalueert op reeks van taken (retrieval, clustering, classificatie, reranking). Hoewel MTEB nuttig kan zijn bij de initiële selectie van het embedding‑model, stellen \textcite{Tang2024} vast dat prestaties van embedding-modellen op MTEB niet overeenkomen met een benchmark die ontworpen is voor domeinspecifieke datasets.

Voor RAG‑systemen is het probleem nog complexer, aangezien de evaluatie zowel rekening moet houden met de kwaliteit van de opgehaalde context (retrieval) als van de getrouwheid (faithfulness) van de gegenereerde tekst. \textcite{Yu2024} tonen in hun survey dat bestaande benchmarks en prestatie-indicatoren onvoldoende zijn om deze interactie te vangen, en pleiten voor een Unified Evaluation Process waarin retrieval‑ en generatie‑metrics gecombineerd worden. Dit heeft geleid tot de ontwikkeling van RAG‑specifieke evaluatieframeworks zoals RAGAS en ARES, die expliciete evaluatie metrics definiëren voor de veelzijdige componenten en processen binnen RAG.

Voor de PoC van Turtle Srl zijn klassieke benchmarks (BLEU, ROUGE, MTEB) daarom niet bruikbaar. Zaken zoals: de mate waarin antwoorden gegrond zijn in interne ESG‑documenten, of alle noodzakelijke informatie effectief wordt opgehaald, en of bronvermelding correct en volledig is worden niet geëvalueerd. Een domeinspecifiek RAG‑evaluatiekader, geïnspireerd door onder meer \textcite{Yu2024}, is dus noodzakelijk.

\subsection{Het RAGAS framework}

De academische standaard voor het kwantificeren van RAG-prestaties is het RAGAS framework. RAGAS, ontwikkeld door \textcite{Es2024}, introduceert metrieken die de "RAG Triad" (Query, Context, Answer) evalueren. De interactie tussen de query, de opgehaalde context en het gegenereerde antwoord worden geëvalueerd met 4 metrieken. Het framework gebruikt LLM-as-a-judge, een krachtige LLM die de output van een andere LLM evalueert, waardoor menselijke tussenkomst overbodig wordt.

De metrieken zijn onderverdeeld in 2 categorieën op vlak van welk component ze evalueren. De generator wordt beoordeeld op faithfulness en answer relevancy en de op vlak van context precision en context recall.

\begin{figure}
  \centering
  \includegraphics[width=0.8\textwidth]{ragas.png}
  \caption[Voorbeeld figuur.]{\label{fig:grail}Overzicht van de implementatie en interpretatie van RAGAS-metrieken binnen het RAGAS framework. Bron: \autocite{Kaarthick2024Ragas}}
\end{figure}

Faithfulness, of getrouwheid, meet of het gegenereerde antwoord uitsluitend gebaseerd is op de opgehaalde context. Dit is cruciaal om hallucinaties op te sporen: elk cijfer in een ESG-rapport moet immers herleidbaar zijn naar de bron. Answer Relevancy beoordeelt in welke mate het antwoord daadwerkelijk een oplossing biedt voor de gestelde vraag, zonder overbodige informatie. Context Precision evalueert of de meest relevante documentfragmenten bovenaan de zoekresultaten stonden. Voor ESG-audits is dit essentieel om de efficiëntie van de retriever te bepalen. Context Recall controleert of alle benodigde informatie om de vraag te beantwoorden daadwerkelijk aanwezig was in de opgehaalde fragmenten \autocite{Dong2025,Roychowdhury2024}.

%\begin{figure}
%  \centering
%  \includegraphics[width=0.8\textwidth]{grail.jpg}
%  \caption[Voorbeeld figuur.]{\label{fig:grail}Voorbeeld van invoegen van een figuur. Zorg altijd voor een uitgebreid bijschrift dat de figuur volledig beschrijft zonder in de tekst te moeten gaan zoeken. Vergeet ook je bronvermelding niet!}
%\end{figure}
%
%\begin{listing}
%  \begin{minted}{python}
%    import os
%    import llm
%    from hypster import HP, instantiate
%
%    def llm_config(hp: HP):
%    model_name = hp.select(["gpt-4o-mini", "gpt-4o"], name="model_name")
%    temperature = hp.float(0.0, name="temperature", min=0.0, max=1.0)
%    max_tokens = hp.int(256, name="max_tokens", max=2048)
%
%    return {
%        "model_name": model_name,
%        "temperature": temperature,
%        "max_tokens": max_tokens
%    }
%
%  \end{minted}
%  \caption[Tekst in lijst code fragmenten]{Tekst onder code blok}
%\end{listing}
%
%\lipsum[7-20]
%
%\begin{table}
%  \centering
%  \begin{tabular}{lcr}
%    \toprule
%    \textbf{Kolom 1} & \textbf{Kolom 2} & \textbf{Kolom 3} \\
%    $\alpha$         & $\beta$          & $\gamma$         \\
%    \midrule
%    A                & 10.230           & a                \\
%    B                & 45.678           & b                \\
%    C                & 99.987           & c                \\
%    \bottomrule
%  \end{tabular}
%  \caption[Voorbeeld tabel]{\label{tab:example}Voorbeeld van een tabel.}
%\end{table}
%%=============================================================================
%% Methodologie
%%=============================================================================

\chapter{\IfLanguageName{dutch}{Methodologie}{Methodology}}%
\label{ch:methodologie}

%% TODO: In dit hoofstuk geef je een korte toelichting over hoe je te werk bent
%% gegaan. Verdeel je onderzoek in grote fasen, en licht in elke fase toe wat
%% de doelstelling was, welke deliverables daar uit gekomen zijn, en welke
%% onderzoeksmethoden je daarbij toegepast hebt. Verantwoord waarom je
%% op deze manier te werk gegaan bent.
%% 
%% Voorbeelden van zulke fasen zijn: literatuurstudie, opstellen van een
%% requirements-analyse, opstellen long-list (bij vergelijkende studie),
%% selectie van geschikte tools (bij vergelijkende studie, "short-list"),
%% opzetten testopstelling/PoC, uitvoeren testen en verzamelen
%% van resultaten, analyse van resultaten, ...
%%
%% !!!!! LET OP !!!!!
%%
%% Het is uitdrukkelijk NIET de bedoeling dat je het grootste deel van de corpus
%% van je bachelorproef in dit hoofstuk verwerkt! Dit hoofdstuk is eerder een
%% kort overzicht van je plan van aanpak.
%%
%% Maak voor elke fase (behalve het literatuuronderzoek) een NIEUW HOOFDSTUK aan
%% en geef het een gepaste titel.

De methodologie van dit onderzoek is zodanig opgesteld om zowel wetenschappelijke geldigheid als praktische toepasbaarheid voor Turtle Srl te waarborgen. Het centrale doel is het systematisch ontwerpen, opzetten en evalueren van een dynamische RAG pipeline die het mogelijk maakt om de  variabele componenten (chunking strategieën, embedding modellen en LLM’s) te testen op één vaste ESG vragenlijst, zonder voor elke mogelijke combinatie een aparte pipeline te moeten bouwen. De gekozen aanpak volgt een sequentieel maar iteratief raamwerk in zes fases, waarbij elke fase deliverables oplevert aan de hand van de doelstellingen en gekozen onderzoeksmethoden. Om de reproduceerbaarheid, repliceerbaarheid en herbruikbaarheid van dit onderzoek te waarborgen, wordt per fase gedocumenteerd welke beslissingen genomen worden en hoe data en configuraties vastgelegd worden.

\section{Fase 1: Requirements verzamelen}

In de eerste fase staat het verzamelen en verduidelijken van requirements centraal. Het doel is om een beeld te krijgen van de noden en eisen van Turtle Srl omtrent ESG rapportage en de criteria waaraan de beoogde optimale configuratie moet voldoen. Daarvoor worden semigestructureerde interviews afgenomen met belanghebbenden van het bedrijf (bijvoorbeeld ESG analisten, data engineers en management) om zowel functionele als niet functionele vereisten in kaart te brengen. Functionele vereisten omvatten onder meer de nood aan bronvermelding, de dekking van specifieke ESRS standaarden en de gewenste vraagtypes in de golden set. Niet functionele vereisten hebben betrekking op criteria die relevant zijn voor de business waarde, zoals latency, kosten per query en dataprivacy.

Parallel aan deze interviews wordt een literatuuronderzoek uitgevoerd naar RAG architecturen, chunking strategieën, embedding modellen, evaluatie frameworks (zoals RAGAS, ARES) en domeinspecifieke toepassingen in ESG  en duurzaamheidsrapportage. Op basis van deze twee informatiestromen wordt een geprioriteerde lijst van requirements opgesteld met behulp van de MoSCoW-methode , aangevuld met een eerste “long list” van mogelijke interessante alternatieven voor de variabele componenten in de RAG pipeline. Deze artefacten vormen de basis voor de daaropvolgende selectie  en ontwerpfases.

\section{Fase 2: Opstellen short list}

In de tweede fase wordt de long list van mogelijke alternatieven herwerkt tot een beheersbare short list van de meest veelbelovende opties per variabel component. Het doel is om, op basis van de in fase 1 gedefinieerde requirements, een onderbouwde selectie te maken van de interessantste embedding modellen, chunking strategieën en LLM’s voor verdere experimentele evaluatie.

Gedurende deze systematische vergelijking worden de belangrijkste eigenschappen en prestaties per alternatief samengebracht in een samenvattende tabel. Voor embedding modellen kan dit onder meer MTEB scores, meertalige recall, vector dimensionaliteit en inference kosten omvatten; voor LLM’s bijvoorbeeld contextlengte, meertalige ondersteuning, faithfulness scores uit de literatuur en kosten per token. Elk alternatief wordt gescoord tegen de requirements (bijvoorbeeld bronvermelding, meertaligheid, ESG relevantie) en reeds bestaande literatuur, wat resulteert in een shortlist per component.

Deze aanpak levert een short list op van de mogelijke technologieën voor de variabele componenten, aangevuld een vergelijkende tabel. Op basis van deze informatie kan bepaalde worden welke modellen of strategieën wel of niet toepasbaar zijn voor de experimentele opzet.

\section{Fase 3: Input data en golden set voorbereiden}

Fase 3 richt zich op de voorbereiding van de domeinspecifieke inputdata en het samenstellen van een “golden set” van vragen voor de latere evaluatie. Het doel is om enerzijds een representatieve dataset aan bedrijfsdocumenten te verzamelen die de interne ESG informatie van Turtle Srl weergeeft, en anderzijds een gestandaardiseerde vragenlijst op te stellen waarmee de RAG configuraties in fase 5 worden getest.

Via bijkomende interviews met domeinexperten wordt bepaald welke documenten nodig zijn om een dataset te verkrijgen die bruikbaar is voor ESG rapportage. Deze data kan bestaan uit jaarverslagen, duurzaamheidsrapporten, beleidsdocumenten of interne richtlijnen. Indien nodig worden documenten geanonimiseerd voordat ze in de experimentele omgeving worden opgenomen om de veiligheid van de data te verzekeren.

Daarnaast wordt een golden set van vragen opgesteld. Deze set omvat ESG-gerelateerde vragen, zoals questionnaires van Ecovadis en VSME, om een vragenlijst te bekomen die representatief is voor Turtle Srl. Voor elke vraag wordt waar mogelijk een referentie antwoord, bijbehorende brondocumenten en relevante pagina  of sectienummers vastgelegd. Deze metadata maakt het mogelijk om zowel automatische evaluatie, met RAGAS, als handmatige expert evaluatie (bijvoorbeeld audits) uit te voeren.

Deze fase resulteert in een gestructureerde verzameling van contextdata die in de vector database zal worden geïndexeerd, en een vragenlijst die als vaste benchmark dient voor alle RAG configuraties.

\section{Fase 4: Ontwikkeling van dynamische RAG-pipeline}
In de vierde fase wordt de RAG pipeline ontworpen en geïmplementeerd. De centrale doelstelling is een dynamische proefopstelling te bouwen die toelaat om de variabele componenten (zoals bepaald in fase 2) op een efficiënte manier te combineren en te testen. Zonder een nieuwe pipeline te moeten maken voor iedere combinatie van de componenten.

Technisch wordt gekozen voor een component gebaseerde architectuur waarin modules voor documentconversie, chunking, embedding, vectoropslag, retrieval en generatie als afzonderlijke bouwstenen worden gedefinieerd. De configuratielaag, in JSON, bepaalt per experimentele run welke chunking strategie, welk embedding model en welk LLM worden gebruikt. Dit maakt het mogelijk om de alle combinaties van de short list systematisch te doorlopen.

Belangrijke aandachtspunten bij de implementatie zijn: meertalige ondersteuning (om zowel Engelstalige als anderstalige ESG documenten te kunnen verwerken), bronvermelding via metadata payloads in de vector database, en logging voorzien van alle relevante runtime informatie (latency, tokenverbruik, gebruikte configuratie, RAG metrics). Op het einde van deze fase wordt een werkende, dynamische RAG pipeline opgeleverd die conform is met de technische vereisten van het bedrijf.

\section{Fase 5: Pipeline testen}

In fase 5 wordt de pipeline getest, met als doel om voor alle geselecteerde configuraties resultaten te verzamelen op basis van de metrics die in fase 1 als requirements zijn gedefinieerd.

De onderzoeksmethode bestaat uit gecontroleerde runs van de pipeline. De golden set vragenlijst uit fase 3 wordt meerdere keren door de pipeline geleid, telkens met een andere configuratie van chunking strategie, embedding model en LLM. Voor elke combinatie worden zowel inhoudelijke metrics (bijvoorbeeld RAGAS scores voor faithfulness, answer relevancy, context precision en context recall) als operationele metrics (latency per vraag, kosten per query, geheugenverbruik) geregistreerd. Daarnaast kunnen steekproeven van antwoorden door ESG experten worden beoordeeld om de automatische evaluatie te valideren.

De resultaten worden opgeslagen in een Jupyter Notebook zodat ze in de volgende fase op een gestructureerde manier kunnen worden geanalyseerd. Uit deze fase komt een volledige set aan experimentele resultaten die toelaat om de verschillende RAG configuraties onderling te vergelijken op basis van dezelfde vragen en evaluatiemetrics.

\section{Fase 6: Resultaten verwerken en conclusie trekken}

De laatste fase richt zich op het verwerken, analyseren en interpreteren van de verzamelde resultaten en om te zetten in duidelijke inzichten voor de PoC van Turtle Srl.

De resultaten uit fase 5 worden visueel voorgesteld in de vorm van grafieken en tabellen die per metric en per component (chunker, embedder, LLM) de prestaties weergeven. Hierbij wordt gezocht naar patronen, zoals configuraties die consequent hoge faithfulness  en answer relevancy scores hebben met lage latency en kosten. Waar relevant worden trade offs in kaart gebracht in overleg met belanghebbende van het bedrijf (bijvoorbeeld hogere nauwkeurigheid ten koste van hogere kosten of langere responstijden).

Op basis van deze analyse wordt een decision tree opgesteld, waarin voor verschillende gebruiksscenario’s (bijvoorbeeld kosten sensitief versus kwaliteits gedreven) de meest geschikte combinatie van variabele componenten wordt voorgesteld. Deze beslissingsstructuur vormt samen met een samenvattende aanbeveling over de “beste” configuratie het resultaat van dit onderzoek.

\lipsum[21-25]

% Voeg hier je eigen hoofdstukken toe die de ``corpus'' van je bachelorproef
% vormen. De structuur en titels hangen af van je eigen onderzoek. Je kan bv.
% elke fase in je onderzoek in een apart hoofdstuk bespreken.

%\input{...}
%\input{...}
%...

\input{conclusie}

%---------- Bijlagen -----------------------------------------------------------

\appendix

\chapter{Onderzoeksvoorstel}

Het onderwerp van deze bachelorproef is gebaseerd op een onderzoeksvoorstel dat vooraf werd beoordeeld door de promotor. Dat voorstel is opgenomen in deze bijlage.

%% TODO: 
\section*{Samenvatting}

% Kopieer en plak hier de samenvatting (abstract) van je onderzoeksvoorstel.
De afgelopen jaren neemt de regulering rond duurzaamheid en emissierapportage toe. Een voorbeeld is de Corporate Sustainability Reporting Directive (CSRD), die bedrijven verplicht om hun milieu-impact transparant en digitaal te rapporteren. De huidige emissiedata in ESG-rapporten is vaak heterogeen, onnauwkeurig en moeilijk automatisch te verwerken, wat de ontwikkeling van betrouwbare, geautomatiseerde rapportagesystemen moeilijk maakt. Tegelijkertijd bieden Retrieval-Augmented Generation (RAG)-architecturen, waarin een LLM wordt gekoppeld aan een externe kennisbron, nieuwe mogelijkheden om domeinspecifieke documentaten te doorzoeken en accurate antwoorden te genereren. Binnen deze architectuur is de keuze van het embeddingmodel cruciaal, omdat dit model bepaalt welke documentcontext wordt opgehaald en dus een directe impact heeft op nauwkeurigheid, betrouwbaarheid en efficiëntie. In deze bachelorproef wordt, vertrekkend vanuit een concrete casus bij Turtle Srl, onderzocht: “Hoe kan de keuze van het embeddingmodel voor een RAG-architectuur helpen bij het identificeren van en rapporteren over de uitstoot van productiebedrijven?”. Het doel is om een proof-of-concept te ontwikkelen met een constante RAG-architectuur en drie variabele embeddingmodellen. Deze architecturen worden getest op een vaste dataset van duurzaamheids- en uitstootrapporten met behulp van een vragenlijst van 30 domeinspecifieke vragen. De prestaties van de embeddingmodellen en het RAG-systeem worden geëvalueerd via RAGAS- en MTEB-metrics. Het onderzoek bevat een literatuurstudie, requirementsanalyse, een experimentele opzet, simulaties van de verschillende embeddingmodellen en een vergelijkende analyse van de prestaties. Er wordt verwacht dat de modellen meetbare verschillen zullen tonen op vlak van context recall, precisie, faithfulness en latency, wat toelaat om sterktes en beperkingen per model te identificeren en te koppelen aan praktijkcriteria zoals kosten, snelheid en compliance-risico. Het resultaat is een beslissingskader in de vorm van een flowchart waarmee productiebedrijven, en in het bijzonder Turtle Srl en haar klanten, onderbouwd kunnen kiezen welk embeddingmodel het best past bij hun RAG-architectuur voor ESG- en emissierapportage, en zo zowel de kwaliteit als de efficiëntie van hun rapportageprocessen verbeteren.

% Verwijzing naar het bestand met de inhoud van het onderzoeksvoorstel
%---------- Inleiding ---------------------------------------------------
\section{Inleiding}
\label{sec:inleiding}

Accountancy Europe wijst erop dat de urgentie rond klimaatverandering en duurzaamheid, bedrijven wereldwijd onder toenemende druk zet om hun milieu-impact transparant te rapporteren: ``Bedrijven zullen duurzaamheidsinformatie in hun managementverslag moeten openbaar maken volgens verplichte Europese normen voor duurzaamheidsverslaglegging en deze in een digitaal, machinaal leesbaar formaat moeten indienen'' \autocite{AccountancyEurope2024}. Deze Europese richtlijn genaamd de Corporate Sustainability Reporting Directive (CSRD), werd van kracht op 5 januari 2023 van kracht werd. \'{E}\'{e}n van de vereisten van de CSRD is dat gerapporteerde duurzaamheidsinformatie extern moet worden geaudit door onafhankelijke auditors, wat de nood voor geautomatiseerde en betrouwbare rapportagesystemen benadrukt \autocite{EuropeanCommission2024}.
\textcite{Nguyen2023} onderzochten Scope 3-emissiedata van verschillende externe leveranciers en documenteerden in \emph{PLOS Climate} dat zelfs bij leveranciers met een hoge correlatie (0.95) slechts 68\% van de individuele datapunten identiek was. Deze inconsistenties zijn te wijten aan verschillen in berekenings\-me\-tho\-do\-lo\-gie\"en, data-bronnen (primair versus secundair), temporele aggregatie (kalenderjaar versus boekjaar) en toewijzingsregels voor gedeelde faciliteiten \autocite{Nguyen2023}.
Tegelijkertijd heeft de opkomst van Retrieval-Augmented Generation (RAG) nieuwe mogelijkheden gecre\"eerd voor het geautomatiseerd verwerken en analyseren van complexe, domeinspecifieke documentatie. \textcite{Lewis2020} introduceerden het Retrieval-Augmented Generation (RAG)-framework. Een AI-framework dat Large Language Models (LLM's) koppelt aan externe, actuele databank. Door domeinspecifieke documentcontext op te halen en te gebruiken, maakt RAG het mogelijk om antwoorden te genereren die zowel feitelijk correct zijn als traceerbaar naar de bron.
Binnen de RAG-architectuur is de keuze van het embeddingmodel cruciaal. Het embeddingmodel vertaalt zowel de gebruikersvraag als de brondocumenten naar numerieke vectoren, die de semantische gelijkenis bepalen en daarmee bepalen welke context wordt opgehaald (retrieved). Cruciaal onderzoek toont aan dat de prestaties van een RAG-systeem significant worden beïnvloed door deze keuze: een kleiner LLM gekoppeld aan het juiste embeddingmodel kan grotere modellen overtreffen \autocite{Canale2025}. Dit benadrukt dat de optimalisatie van het retrieval-mechanisme – de taak van het embeddingmodel – essentieel is voor de algehele betrouwbaarheid, nauwkeurigheid en kosteneffici\"entie van het systeem.

De concrete probleemstelling van dit onderzoek wordt benaderd vanuit een casus van het bedrijf Turtle Srl, een softwarebedrijf dat zich inzet om bedrijven te ondersteunen in hun groei, met een focus op de digitalisering en duurzaamheid van industri\"ele processen \autocite{TurtleSrl2023}.
Het bedrijf is van plan om een Proof of Concept (PoC) te ontwikkelen dat binnen een bedrijf kan worden toegepast: een chatbot die interne, domein specifieke vragen kan beantwoorden. Het is hierbij essentieel dat het systeem duidelijk aangeeft welke documenten het heeft gebruikt om elk antwoord te genereren.
Hoewel diverse technologische keuzes (chunkingstrategie\"en, embeddingmodellen, en Large Language Models (LLM's)) de prestaties van het systeem beïnvloeden, is de keuze van het embeddingmodel cruciaal voor de relevantie van de opgehaalde documentcontext. Voor Turtle is er echter geen duidelijk kader over hoe de selectie en configuratie van zowel lokaal gehoste als propri\"etaire embeddingmodellen de kwaliteit en effici\"entie van de RAG-prestaties direct beïnvloedt.


Dit onderzoek vertrekt vanuit de volgende onderzoeksvraag:

\textit{Hoe kan de keuze van het embeddingmodel voor een RAG-architectuur helpen bij het identificeren van en rapporteren over de uitstoot van productiebedrijven?}

Om deze hoofdvraag te beantwoorden, worden de volgende deelvragen onderzocht:

\begin{itemize}
  \item In hoeverre vari\"{e}ren de dataformaten en terminologie in de documenten van verschillende productiebedrijven?
  \item Welke criteria (zoals snelheid, nauwkeurigheid, recall, en kosteneffectiviteit) moeten worden gebruikt voor de functionele evaluatie van de RAG-output om te bepalen of een antwoord als 'succesvol' kan worden beschouwd binnen het ESG-rapportageproces?
  \item Welke gangbare testen worden uitgevoerd om de resultaten te evalueren (zoals hallucinatie, contextuele verschuiving, en data noise)?
  \item Welke karakteristieken (model-grootte, dimensionaliteit, training-data, multilingualiteit) onderscheiden embeddingmodellen van elkaar?
  \item Wat zijn de prestatieverschillen tussen verschillende embeddingmodellen (bijvoorbeeld BGE-M3, text-embedding-3-large, multilingual-e5-large) bij het genereren van vectorrepresentaties voor domeinspecifieke ESG-emissieteksten?
  \item Hoe beïnvloedt de keuze van het embeddingmodel de RAGAS-metrics (Retrieval-Augmented Generation Assessment Suite: faithfulness, context precision, context recall, answer relevancy) in een gecontroleerde en constante RAG-architectuur?
\end{itemize}

De beoogde doelgroep bestaat uit bedrijven in de productiesector die de stap willen zetten naar de implementatie van een RAG-systeem. Dit onderzoek focust zich specifiek op RAG-systemen die gebruikt worden voor het analyseren, verwerken en rapporteren van de emissiedata van productiebedrijven. 

Het doel van dit onderzoek is om productiebedrijven tot een gefundeerde beslissing te laten komen over de keuze van het embeddingmodel dat wordt gebruikt in hun RAG-architectuur. 

Voor dit onderzoek wordt een proof of concept uitgewerkt waarin een constante RAG-architectuur met variabele embeddingmodellen zal worden getest op een (vaste) dataset (voorzien door Turtle Srl). De dataset bestaat uit duurzaamheids- en uitstootrapporten van meerdere productiebedrijven. Elke variant van de RAG-architectuur zal moeten antwoorden op een reeks van 30 vragen. Hierbij wordt zowel de volledige output van het RAG-systeem ge\"evalueerd (m.b.v. het RAGAS framework), alsook de prestaties van de verschillende embeddingmodellen (m.b.v. de MTEB framework). Deze meetwaarden zijn cruciaal om een correct aanbeveling te doen bij het kiezen van het embeddingmodel voor een RAG-systeem.

Na deze inleiding volgt een literatuurstudie die de theoretische fundamenten\\van RAG-architecturen \textcite{Lewis2020}, embeddingmodellen \textcite{Muennighoff2022} \textcite{Chen2024}, de evaluatiemethoden en -criteria voor embeddingmodellen en ESG-rapportage \autocite{GGP2013} uiteenzet. Daarna de gevolgde methodologie en experimentele opzet, gebaseerd op de evaluatieframeworks van \autocite{Es2024} en \autocite{Muennighoff2022} voor de PoC. Vervolgens komen de verwachte resultaten aan bod. Tot slot wordt een overzicht gegeven van de conclusies en worden aanbevelingen geformuleerd voor zowel verder onderzoek als praktische implementatie.

%---------- Stand van zaken ---------------------------------------------------

\section{Literatuurstudie}
\label{sec:literatuurstudie}

Deze literatuurstudie een overzicht van de huidige stand van zaken voor dit onderzoek. Het probleemdomein betreft de uitdagingen waarmee productiebedrijven geconfronteerd worden bij het identificeren en rapporteren van hun uitstoot in het kader van ESG-compliance, met specifieke aandacht voor de Corporate Sustainability Reporting Directive (CSRD) en European Sustainability Reporting Standards (ESRS). Het oplossingsdomein focust op Retrieval-Augmented Generation (RAG) architecturen, embeddingmodellen en welke invloed de keuze van het embeddingmodel kan hebben op de resultaten van het RAG-systeem.

\subsection{ESG-rapportage en de Environmental Pilaar}

Environmental, Social en Governance (ESG)-rapportage heeft zich de afgelopen jaren ontwikkeld van een vrijwillige praktijk tot een verplichting voor bedrijven wereldwijd \autocite{UNGC}. De Corporate Sustainability Reporting Directive (CSRD), die op 5 januari 2023 van kracht werd, vervangt zijn voorganger nl. de Non-Financial Reporting Directive (NFRD) en breidt het aantal rapportageplichtige bedrijven uit van 12.000 naar 42.500 \autocite{AccountancyEurope2024}. Voor organisaties betekent de CSRD dat zij gedetailleerde informatie moeten rapporteren over hun milieu-impact conform met de European Sustainability Reporting Standards (ESRS), waarbij de rapportage in een digitaal, machinaal leesbaar formaat (ESEF) moet worden ingediend en extern geaudit door onafhankelijke auditors \autocite{EuropeanCommission2024}.

Binnen het ESG-kader vormt de Environmental pilaar een kritisch aandachtsgebied, met specifieke focus op broeikasgasemissies (Scope 1, 2 en 3), energieverbruik, waterbeheer en circulariteit \autocite{AccountancyEurope2024}. Het Greenhouse Gas Protocol (GHG Protocol), een gestandaardiseerd kader voor het meten en beheren van emissies over de volledige waardeketen \autocite{GGP2013}, categoriseert emissies als volgt: Scope 1 omvat directe emissies van bronnen die het bedrijf bezit of controleert (bijvoorbeeld brandstofverbruik in bedrijfswagens, procesemissies); Scope 2 zijn alle indirecte emissies van aangekochte energie (elektriciteit, warmte); en Scope 3 omvat alle andere indirecte emissies in de waardeketen, zowel upstream (leveranciers, transport) als downstream (gebruik van verkochte producten, afvalverwerking) \autocite{Wiedmann2020}. 

\subsection{Variatie in Dataformaten en Terminologie van Emissierapporten}

Een bewezen uitdaging bij het automatiseren van ESG-dataverwerking is de heterogeniteit van emissiedata in domeinspecifieke uitstootdocumenten. \textcite{Nguyen2023} vonden in hun onderzoek naar Scope 3-emissiedata van meerdere externe gegevensleveranciers duidelijk verschillen tussen de gerapporteerde emissiewaarden van dataproviders. Uit het onderzoek bleek dat de emissiewaarden van een groep leveranciers, desondanks correlatiewaarden van 95\%, slechts 68\% identieke gegevenspunten hebben \autocite{Nguyen2023}. Deze inconsistentie maakt geautomatiseerde extractie zonder uitgebreide domeinkennis een grotere uitdaging. Deze inconsistenties zijn het resultaat van verschillen in berekenings-me-tho-do-lo-gie"en, data-bronnen (primair versus secundair), temporele aggregatie (kalenderjaar versus boekjaar) en toewijzingsregels \autocite{Nguyen2023}. Het NGFS (Network for Greening the Financial System) benadrukt in een onderzoek naar het verbeteren van de gerapporteerde emissiedata dat de grote heterogeniteit in benaderingen van verschillende dataproviders, vooral voor Scope 3-emissies, de vergelijkbaarheid bemoeilijkt \autocite{NGFS2024}.

Daarnaast presenteren duurzaamheidsrapporten informatie in zeer diverse formaten: plain text, tabellen, grafieken en afbeeldingen, wat geautomatiseerde Information Extraction (IE) significant bemoeilijkt \autocite{Morales2022}. Deze fragmentatie van ESG-data maakt gestandaardiseerde rapportage complex en ineffici\"ent. PwC Luxembourg rapporteerde dat 55\% van de bedrijven uitdagingen verwacht op het gebied van datakwaliteit en consistentie bij CSRD-rapportage, en 45\% van de bedrijven zijn bezorgd over het feit of zij voldoende resources hebben om aan de richtlijnen te voldoen \autocite{FundsEurope2025}.

\subsection{Retrieval-Augmented Generation (RAG)}

Om deze uitdagingen het hoofd te bieden, biedt Retrieval-Augmented Generation (RAG) een veelbelovend architectuurpatroon.\\Geïntroduceerd door \autocite{Lewis2020}, het kernidee achter RAG is dat LLMs, hoewel krachtig in het vastleggen van taalpatronen en algemene kennis tijdens pre-training, beperkt zijn in hun vermogen om actuele, domeinspecifieke of feitelijke informatie te onthouden en correct toe te passen. Door een externe kennisbron combineren met een LLM, kunnen RAG-systemen relevante documenten ophalen en deze gebruiken als context voor het genereren van antwoorden \autocite{Lewis2020}.

De toepassing van RAG-systemen voor ESG-rapportage werd onderzocht door verschillende recente studies. ESGReveal, geïntroduceerd door \textcite{Zou2023} in december 2023, is een methode voor de systematische extractie en analyse van Environmental, Social, and Governance data uit bedrijfsrapporten. Het systeem gebruikt LLMs gecombineerd met RAG-technieken en omvat drie modules: een ESG metadata-module voor criteria-queries, een rapport-preprocessing-module voor het bouwen van databases, en een LLM-agent-module voor data-extractie \autocite{Zou2023}.

\subsection{Embeddingmodellen: Karakteristieken en Eigenschappen}

Een embeddingmodel is verantwoordelijk voor het omzetten van data uit de externe databank in een vectorrepresentaties genaamd \textit{embeddings} \autocite{Khanna2025}. Deze embeddings vormen de kern van het RAG-retrieval mechanisme door semantische zoekopdrachten in vector databases mogelijk te maken. Embeddingmodellen onderscheiden zich op basis van meerdere karakteristieken: model grootte, de embedding dimensionaliteit, domeinspecificiteit en multilingualiteit.

\subsection{Evaluatie van Embeddingmodellen: MTEB Framework}

Om embeddingmodellen systematisch te benchmarken, ontwikkelden \textcite{Muennighoff2022} het Massive Text Embedding Benchmark (MTEB), een publiek framework voor text embedding evaluatie \autocite{Muennighoff2022}. MTEB omvat 58 datasets verspreid over 8 embedding taken: Classification, Clustering, Pair Classification, Reranking, Retrieval, Semantic Textual Similarity (STS), Summarization en BitextMining in 112 talen \autocite{Muennighoff2022}. Voor RAG-toepassingen is vooral de Retrieval taak relevant, waarbij documenten worden gerankt op basis van hun relevantie voor queries en gescoord via metrics zoals NDCG@10 (Normalized Discounted Cumulative Gain at rank 10) \autocite{Muennighoff2022}.

\subsection{Invloed van Embeddingmodel op RAG-prestaties}

De keuze van het embeddingmodel heeft een fundamentele impact op RAG-prestaties, met effecten die meetbaar zijn via zowel RAGAS- als MTEB-metrics. \textcite{Canale2025} introduceerden BES4RAG, een framework specifiek ontworpen om embeddingmodellen te evalueren op basis van question-answering accuracy in plaats van standaard retrieval metrics \autocite{Canale2025}. Hun experimentele resultaten op drie diverse datasets tonen aan dat:

Dataset-afhankelijkheid: De optimale embeddingkeuze varieert significant per dataset, wat het belang benadrukt van dataset-specifieke selectie \autocite{Canale2025}. Verschillende datasets leveren fundamenteel verschillende optimale embeddings op, wat one-size-fits-all benaderingen ineffectief maakt.

Embedding kwaliteit versus model grootte: Een kleiner LLM gekoppeld aan een effectief embeddingmodel kan betere prestaties leveren dan een groter LLM met zwakkere embeddings \autocite{Canale2025}. Dit suggereert dat investering in hoogwaardige embeddings vaak effectiever is dan het upgraden naar grotere generative modellen.

Voor emissiedata specifiek toonden \textcite{Zhao2025} aan dat knowledge graph embeddings een perfecte accuracy (F1=1.0) behaalden op emission factor retrieval queries, terwijl text-only embeddings een veel lagere accuracy behaalden (F1<0.08) \autocite{Zhao2025}. Dit benadrukt dat domeinspecifieke embedding benaderingen essentieel zijn voor numerieke precisie-vereisten in ESG-rapportage, waar kleine fouten in emission factor retrieval kunnen leiden tot compliance violations en financi\"ele misstatements.

%---------- Methodologie ------------------------------------------------------
\section{Methodologie}
\label{sec:methodologie}

Dit onderzoek wordt uitgevoerd aan de hand van een vergelijkende studie waarin de invloed van verschillende embeddingmodellen op de kwaliteit van een RAG-systeem wordt geanalyseerd. Het onderzoek volgt een iteratief proces, dat bestaat uit zes fasen, met als doel het opleveren van een beslissingskader voor het kiezen van het geschikte embeddingmodel.

\paragraph{Literatuurstudie \& requirementsanalyse (Week 1-3)}

In de literatuurstudie worden de werking van RAG en embeddingmodellen is uiteengezet. In deze fase wordt de vertaalslag gemaakt naar de praktijk d.m.v. interviews met werknemers van het bedrijf Turtle Srl. Tijdens deze contactmomenten worden de functionele requirements vastgelegd en wordt bepaald welke specifieke kwaliteitsmetrics (RAGAS, MTEB, BES4RAG) prioriteit hebben voor hun casus.

\paragraph{Experimentele voorbereiding (Week 4-5)}\mbox{}\\
Daarna wordt de dataset, aangeleverd door Turtle Srl, voorbereid voor verwerking. Dit proces omvat het selecteren van representatieve ESG-rapporten en het (her)structureren van de\\data (pre-processing). Voor het chunken van de documenten, de dataset verdelen in een gelijkwaardige subset van gegevens (chunks) \autocite{Dozza2013}, wordt gebruik gemaakt van Python libraries zoals \texttt{LangChain} of \texttt{Unstructured}, waarbij een\\vaste chunking-strategie wordt gehanteerd om de variabele invloed te beperken tot enkel het embeddingmodel.

\paragraph{Ontwikkeling RAG-pipeline (Week 6-8)}

In de derde fase wordt de experimentele opstelling opgezet. Er wordt een RAG-pipeline ontwikkeld in Python, waarbij de architectuur het toelaat om het embeddingmodel eenvoudig te wisselen ('hot-swappable'). Het RAG-framework dat wordt gebruikt is LagChain. De infrastructuur wordt ondersteund door Qdrant als de vector database, die in een Docker-container draait om de vereiste hoge performantie en effectieve ondersteuning voor verschillende vector-dimensies te garanderen. Binnen deze opstelling worden drie verschillende embeddingmodellen getest ter vergelijking: het open-source model HuggingFace, een zwaarder model van OpenAI (text-embedding-3), en BAAI/bge-m3 dat beter presteert in meertaligheid (Multi-Lingual). Voor de generatie component (de LLM) wordt consistent GPT-4o gebruikt om de vergelijking tussen de embeddingmodellen te waarborgen. Ten slotte wordt voor het draaien van de lokale embeddingmodellen en inferentie gebruikgemaakt van de hardware van Google Colab Pro.

\paragraph{Gecontroleerde testen (Week 9)}\mbox{}\\
Vervolgens wordt de pipeline onderworpen aan een geautomatiseerde testprocedure d.m.v. simulaties. Een vaste 'Golden Set' van 30 vragen, representatief voor de domeinspecifieke aard van emissiedata van productiebedrijven, wordt op elke variant van de pipeline getest. Hierbij wordt gemeten hoe accuraat de opgehaalde context is (retrieval) en hoe goed het antwoord is (generation).

\paragraph{Resultaten verzamelen en interpreteren (Week 10-11)}

De resultaten van de simulaties worden kwantitatief geanalyseerd met behulp van het RAGAS-framework (zoals besproken in de literatuurstudie). De modellen worden vergeleken op metrics zoals \textit{Context Recall}, \textit{Context Precision} en \textit{Faithfulness}. Deze data wordt uiteindelijk gebruikt om te bepalen welk model de beste 'fit' heeft voor specifieke scenario's.

\paragraph{Beslissingskader \& rapportering (Week 12)}

In de laatste fase worden alle bevindingen gebundeld in een visueel ondersteund onderzoeksrapport. De resultaten van beide testgroepen worden afgebeeld aan de hand van grafieken, tabellen en diagrammen die de verschillen in retrieval kwaliteit, efficiëntie en relevantie weergeven.

Het onderzoek wordt afgesloten met een Proof of Concept (PoC) in de vorm van een flowchart. De PoC stelt productiebedrijven in staat om, op basis van hun specifieke randvoorwaarden (budget, snelheid, nauwkeurigheid), een gefundeerde keuze te maken voor een embeddingmodel voor hun RAG-systeem. Tot slot worden de bevindingen gerapporteerd en gekoppeld aan de oorspronkelijke onderzoeksvragen. De planning voor dit onderzoek wordt weergegeven in Figuur~\ref{fig:gantt}.

%---------- Verwachte resultaten ----------------------------------------------
\section{Verwacht resultaat, conclusie}
\label{sec:verwachte_resultaten}

Op basis van de \textit{Massive Text Embedding Benchmark} (MTEB) wordt verwacht dat de prestaties van de onderzochte modellen zullen variëren afhankelijk van de taal en complexiteit van de documentatie. Concreet wordt verwacht dat \textbf{BAAI/bge-m3} de hoogste \textit{Context Recall} zal behalen bij meertalige documentatie, aangezien dit model specifiek is getraind voor 'cross-lingual retrieval' en consistent bovenaan staat in de MTEB-ranglijsten voor meertalige taken. Voor \textbf{OpenAI text-embedding-3-large} wordt een hoge \textit{Context Precision} verwacht vanwege de uitgebreide trainingsdata, maar met een hogere latency (verwerkingstijd) door de afhankelijkheid van API-calls in vergelijking met de lokaal gehoste modellen. \textbf{Multilingual-e5-large} zal naar verwachting fungeren als een sterke baseline die een balans biedt tussen snelheid en accuraatheid.

De resultaten zullen worden gepresenteerd in een vergelijkende tabel waarin de RAGAS-metrics (Recall, Precision, Faithfulness) worden afgezet tegen de computationele kosten en snelheid.

De meerwaarde van dit onderzoek ligt in het verduidelijken van de embeddingmodel keuze voor productiebedrijven die onder de CSRD-wetgeving vallen. Aangezien CSRD-rapportage audits vereist, is de traceerbaarheid (\textit{Faithfulness}) van het antwoord cruciaal.

Dit onderzoek levert een beslissingskader (PoC) op dat bedrijven helpt te bepalen welk embeddingmodel de beste keuze is voor hun RAG-systeem, aan de hand van meerdere criteria. Hiermee kunnen bedrijven voldoen aan de vraag naar transparante emissierapportage zonder onnodige overinvestering in cloud-infrastructuur. 

Er wordt gesteld dat de keuze van het embeddingmodel een significante invloed heeft op de retrieval-kwaliteit. Mocht uit de resultaten blijken dat de verschillen tussen de modellen minimaal zijn (bijvoorbeeld $<5\%$ verschil in RAGAS-scores), dan zou dit de nulhypothese verwerpen en suggereren dat voor deze specifieke casus andere factoren, zoals de kwaliteit van de data-preprocessing of chunking-strategie, dominanter zijn dan het embeddingmodel zelf. Dit zou een belangrijk inzicht zijn voor de praktische implementatie bij Turtle Srl.

%%---------- Andere bijlagen --------------------------------------------------
% TODO: Voeg hier eventuele andere bijlagen toe. Bv. als je deze BP voor de
% tweede keer indient, een overzicht van de verbeteringen t.o.v. het origineel.
%\input{...}

%%---------- Backmatter, referentielijst ---------------------------------------

\backmatter{}

\setlength\bibitemsep{2pt} %% Add Some space between the bibliograpy entries
\printbibliography[heading=bibintoc]

\end{document}
